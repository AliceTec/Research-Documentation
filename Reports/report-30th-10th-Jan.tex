\documentclass[10pt,a4paper]{article}
% usepackages
\usepackage[latin1]{inputenc}
\usepackage[english]{babel}
% math
\usepackage{amsmath}
\usepackage{amsfonts}
\usepackage{amssymb}
\usepackage{mathtools}
\usepackage{latexsym}
% formatting
\usepackage{parskip}
\usepackage{fullpage}
\usepackage{pgf,tikz}
\usepackage{mathrsfs}
\usetikzlibrary{arrows}
\pagestyle{empty}
% graphics
%\usepackage[pdftex]{graphicx}
\usepackage{caption}
\usepackage{subcaption}
%\usepackage[below,section]{placeins} % the one below is better for short assignments
\usepackage{float} % provides H as float placement specifier
% extras
\usepackage[pdftex,a4paper,colorlinks=true,urlcolor=blue]{hyperref}
\urlstyle{same}
\usepackage{moreverb} %\verbatimtabinput{filename.py} preserves indentation

%Numbering first level list roman (i,ii,iii) instead of arabic (1,2,3)
% options are \roman \Roman \alph \Alph \arabic
\renewcommand{\theenumi}{\roman{enumi}} 
\renewcommand{\theenumii}{\roman{enumii}}
\newcommand{\Int}{\int\limits}
% Also achieved with the enumerate package
\usepackage{enumerate}
%\numberwithin{equation}{section}%

% author/title details
\author{Alice NANYANZI (\href{mailto:alicenanyanzi@aims.ac.za}{alicenanyanzi@aims.ac.za})}
% \title{Course Title: Assignment X}
\title{Report: 16th Jan-3rd Feb, 2017}
\begin{document}
\maketitle
\section*{Week 1 (16th Jan-3rd Feb,2017):}

\begin{itemize}
\item Identify if there is a relationship between minimum cut and the laplacian centrality of nodes around the minimum cut.\\\\
Results:\\
Since laplacian centrality is achieved with removal of a node from a network, we therefore used the line graph of the given node (where edges form nodes and edges are incident if they share a node within the original graph). research still on going.
\item Read paper about Robustness in starling flock
Aim was to get to understand how the flock handles robustness despite the external influences. 
\end{itemize}
\begin{itemize}
\item The computation for robustness in this paper raised concerned that is to say I did not understand how the computation is carried out.
\item Seven is a magic number. Each bird maintains interaction among seven other birds irrespective of the distance of separation.
\end{itemize}
\section*{Week 2 (3rd-10th Feb,2017):}
\begin{enumerate}[a)]
\item Leverage centrality:
\begin{itemize}
\item The leverage centrality is a centrality measure for brain networks.The motivation behind this measure is that the relative importance of a node is based on how its immediate neighbours rely on it for information. its derived from degree centrality.
\item A high degree node is not highly central in leverage if its neighbours are also high degree nodes.
\item Leverage centrality does not assume that information flows following shortest path or in a serial manner as compared to other betweenness and closeness centralities.
\item However, question about how the centrality is computed that is division by the degree of the node whose centrality is being calculated.
\end{itemize}
\item Relationship between laplacian energy of a graph and that of its corresponding line graph:\\
So far still working on this.

\item Read about laplacian centrality for directed networks. The out degree is considered in this case.
What would the computations using in degree imply?

\end{enumerate}

\end{document}
